\chapter{Introduction}
\label{chap:intro}

\section{Seqeuncing Technology}
\label{sec:seq}
Since Sanger developed a chain-terminating procedure for DNA sequencing over forty years ago \citep{Sanger1977-lo}, sequencing capabilities have grown, at first steadily, then astronomically \citep{Schatz2013-vw}. The advent of sequencing-by-synthesis ushered in ‘next-generation’ sequencing (NGS) methods, whereby DNA sequencing became ‘massively parallel’ in nature and vastly accessible to researchers in all fields of biology and medicine. These NGS methods typically involve immobilizing millions of DNA fragments, amplifying them, and then observing the activity of DNA polymerase as it synthesizes the complement strands to the amplified fragments. Commonly, this observation is done through imaging fluorescently labeled nucleotides one base addition, or cycle, at a time \citep{Shendure2017-oy}.

The highly democratized nature of NGS has enabled researchers in clinical microbiology to use it for a variety of important applications. These range from cataloging and surveilling genetic determinants of antimicrobial resistance (AMR) \citep{Crofts2017-ni, Canica2019-ho, Toth2020-ov, Thanner2016-wy, Hendriksen2019-qi}, to monitoring outbreaks of infectious diseases \citep{Dipaola2020-bw, Lu2020-ti}, to analyzing entire human microbiomes with metagenomic sequencing \citep{Chiu2019-cg}. Based on the advances brought about by NGS, some have even called for establishing a ‘digital immune system’ whereby sequencing-based microbial surveillance would detect threats of outbreak, which then could be contained before they become too difficult to control \citep{Schatz2012-ow}.

While NGS technology has unlocked enormous advances in clinical microbiology, it is limited by the fragment lengths it can handle and its reliance on amplification. Typical NGS methods are unreliable at sequencing individual DNA fragments multiple thousands of base pairs long \citep{Heather2016-wb}, and can only read stretches of a few hundred nucleotides at a time. These short read lengths make the sequencing data difficult to work with for many downstream applications, including genome assembly and analysis of repetitive genomic loci. Meanwhile, the amplification process obliterates any base modification information, such as methylation, potentially present on the native DNA fragment.

The rise of third generation, single-molecule sequencing just in the past decade has begun to address these shortcomings. These methods, also known as ‘long-read’ sequencing, interrogate individual DNA molecules without the need for amplification, and can read stretches of thousands of nucleotides at a time \citep{Jain2018-qp}. Nanopore sequencing in particular does this by measuring the minute fluctuations in ionic current as a single stranded DNA molecule passes through a transmembrane protein pore. Because no amplification or other chemical treatments are required prior to sequencing, base modification information remains intact and can be read simultaneously with the nucleotide sequence \citep{Simpson2017-wb, McIntyre2017-ed}. Also unlike NGS methods, the current single molecule sequencing procedures are not dependent on imaging single bases from all the reads at once in a synchronized fashion. Data is collected from all pores independently, which enables data streaming to analysis pipelines even as more data are still being collected. To further the role of sequencing for infectious disease applications, I leverage the new capabilities of nanopore sequencing for AMR detection, genome assembly of eukaryotic pathogens, and metagenomics.

\section{Antimicrobial resistance}
\label{sec:amr}
Since Alexander Fleming first observed the ‘bacteriolytic’ properties of a mysterious ‘mould broth filtrate’ which he termed ‘penicillin’ \citep{Fleming1929-cb}, antibiotics have been an unprecedented and miraculous silver bullet against previously deadly bacteria. Usually produced and isolated from fungi \citep{Martinez2008-cf}, antibiotics are small molecules capable of killing bacteria or inhibiting their growth without damaging eukaryotic cells or tissues in the vicinity. Not only are antibiotics used to cure infectious diseases, but they have enabled more and more complex medical interventions such as surgery and chemotherapy by drastically reducing the risk and ramification of bacterial infections \citep{Crofts2017-ni}. Outside of medicine, antimicrobials have also been used extensively in animal agriculture to control disease \citep{Aarestrup2015-zu}, as populations of food animals are scaled to accommodate global diets that continue to demand more animal protein \citep{Van_Boeckel2019-zl}. Consequently, antibiotics are one of the most commonly prescribed classes of drugs in recent decades, and their use is only becoming more widespread \citep{Van_Boeckel2014-io}.

However, for as long as fungi have produced antimicrobials, bacteria have evolved resistances to them \citep{DCosta2011-yn}. As human usage of antibiotics has occurred ubiquitously and without restraint, selection pressures have caused a rapid proliferation of antibiotic resistant strains of bacteria. Even synthetic antibiotics such as quinolones sustained only three decades of widespread usage before resistances began to develop, intensify, and spread \citep{Laxminarayan2013-dk, Ruiz2012-cx}. Multidrug-resistant strains of bacteria have also emerged and become prominent, deepening the international crisis of AMR \citep{Tamma2014-dh}.

In the clinic, the prevalence of AMR makes it difficult to immediately prescribe the most effective interventions for patients colonized with commonly drug resistant bacteria. Not only does this cost potentially crucial time from patient treatment, but it could cause antibiotic waste and contribute to selection pressures causing drug resistances to develop in the first place. In Chapter 2, I explore the use of third generation sequencing in the clinic to rapidly detect drug resistances in order to shorten the time to effective antibiotic therapy and enable antibiotic stewardship.

\section{Genome Assembly}
\label{sec:asm}
High quality, complete genomes are crucial not only for population and comparative genomics, but they also commonly underpin gene expression studies, epigenetics assays, and molecular diagnostics \citep{Rhie2021-xb}. Because highly parallel sequencing technologies cannot record whole genomes on a single read, genomes must be reconstructed out of millions or billions of reads. This process has been likened to a large jigsaw puzzle, where reads must be overlapped, oriented and fit together in order to build the larger picture of the genome \citep{Sohn2018-lf}. Contiguous sequences constructed by overlapping reads in this fashion are known as ‘contigs,’ which typically represent large sections of chromosomes.

Repetitive and low complexity regions of the genome have been difficult to resolve using NGS technologies. The short reads often cannot span these regions, making it difficult to unambiguously determine how long they are, and how many repeats each region contains \citep{Paszkiewicz2010-yf}. Contigs are typically terminated at these ambiguous regions, resulting in highly fragmented assemblies containing thousands of contigs. Long read data capable of spanning repetitive regions are able to resolve the ambiguities they cause, resulting in much more contiguous genome assemblies with longer and fewer contigs. Genome assemblies constructed from only long read data are more prone to single-base errors due to the lower accuracy of the long reads, but NGS data gathered on the same sample can be used to correct most small errors \citep{Goodwin2015-qs}. Using both long-read and NGS data leverages the benefits and addresses the weaknesses of both sequencing technologies.

Only with contiguous, reference-quality genome assemblies can the roles of repeat structures and long, repetitive genes be analyzed. In pathogenic fungi, some adhesion proteins tend to be encoded in long genes with tandem repeats embedded within. These proteins are of particular interest as they are thought to be involved in enabling pathogenicity \citep{Timmermans2018-ci}. In Chapter 3, I use both NGS and long-read data to assemble the genome of Candida nivariensis, a pathogenic yeast, and use the assembled genome to explore the long, repetitive genes encoding adhesions in this species.

\section{Metagenomics}
\label{sec:asm}
Microbes are ubiquitous, and structured microbial communities, or microbiomes, can be found associating with a variety of hosts and environmental niches \citep{Quince2017-ay}. Increasingly, the human associated microbiomes are found to play an important role in human health \citep{Fan2021-hh}. Studying complex communities using traditional microbial methods based on culturing bacteria has been difficult, as not all microbes can be cultured, and the culturing process itself would be likely to alter the composition of the community \citep{Quince2017-ay}.

Community members can be identified and quantified using 16S sequencing, whereby the 16S genes of all microbial genomes in the community are simultaneously amplified, and then sequenced. By comparing only these 16S sequences to each other and to reference databases, operational taxonomic units (OTUs) making up the community can be determined, and taxonomy can be assigned \citep{Johnson2019-wk}. While 16S-based methods are effective for studying the organism composition of microbiomes, it cannot directly shed any light on the functional capabilities of the microbes.

By contrast, metagenomics sequencing captures the full genetic complement of the community, including any genes, plasmids, or phages which may be present in the cells and their environs. While more sequences are captured with metagenomics, analyzing this data becomes more difficult computationally \citep{Breitwieser2019-zp}. One common approach to analysis involves assembling the metagenome in order to determine identity and functions of the microbes in the community \citep{Lapidus2021-dj}. Metagenome assembly is functionally similar to genome assembly of a single organism, where reads are overlapped in order to construct contigs. However, because metagenomic contigs can originate from an undetermined number of organisms, grouping, or ‘binning,’ these contigs according to the species of origin is a crucial yet challenging step in metagenomic analysis \citep{Yue2020-cm}.

Binning metagenomic contigs into metagenome assembled genomes (MAGs), using NGS data has typically been done using a combination of the contigs’ kmer-spectra, and differential coverage \citep{Ghurye2016-mb}. While these methods work well for bacterial chromosomes, they are less effective for binning mobile genetic elements (MGEs), especially if these MGEs are capable of replicating independently of the host chromosome, as most plasmids are. As with genome assembly of a single organism, the use of NGS data in metagenomic assembly limits the lengths of the contigs themselves, potentially resulting in unresolvable repeats and truncated gene sequences.

By applying long-read sequencing to metagenomic analysis, it is possible to assemble much longer contigs from microbiome samples. Furthermore, because native base modification information is preserved, it can be a powerful basis for binning, as modifications are preserved on MGEs and are not affected by chromosome-independent replication. In Chapter 4, I use methylation calls derived from nanopore sequencing for metagenomic binning, and assess its effectiveness.
