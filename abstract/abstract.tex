\chapter*{Abstract}
\label{chap:abstract}
\addcontentsline{toc}{chapter}{Abstract}
\markboth{Abstract}{Abstract}

While next generation sequencing (NGS) has enabled massively parallel DNA sequencing for lower and lower cost, the development of third generation nanopore sequencing offers several key advantages over older sequencing methods. Nanopore sequencers are pocket-sized, making them orders of magnitude cheaper than the next most affordable alternative and the ideal option for wide deployment. They are capable of providing data in real-time, saving valuable hours before data analysis can begin. Additionally, they are able to sequence reads several thousand basepairs long, as opposed to the hundreds of basepairs NGS platforms are capable of, and they embed base modification data without the need for specific treatment beforehand. Given these advantages, in this thesis I examine the application of nanopore sequencing to the study of human pathogens.

First, we use nanopore sequencing to characterize antimicrobial resistance (AMR) in forty clinical isolates. We analyzed real-time data to quickly identify AMR genes, assembled genomes to identify chromosomal mutations, and used short-read sequencing data to correct the errors in the assemblies. With sequencing data, we found that time to effective antibiotic therapy could be shortened by as much as 20 hours compared to standard antimicrobial susceptibility testing (AST).

Second, we leverage the long reads of nanopore sequencing to assemble the genome of a pathogenic yeast, \textit{Candida nivariensis}. Previous efforts to assemble this yeast genome relied solely on short-read NGS data, resulting in a highly fragmented genome. Using nanopore data, we achieve a much higher contiguity, capture previously missing portions of the genome. Furthermore, we demonstrate that our more contiguous genome can be used to better study long and repetative genes, such as those involved in pathogenticity to humans.

Third, we use the base modification information embedded in nanopore sequencing data to call methylation in metagenomic assemblies. These calls enable the binning of metagenomic contigs according to methylation signature without the need to collect additional data. We demonstrate the efficacy of this method on a synthetic community sample, a simple two-bacteria system, and a clinical sample with matched proximity ligation binning data.

These applications of nanopore sequencing demonstrate its potential and its utility for all fronts of pathogen genomics research.
